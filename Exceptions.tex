\section{Exceptions}

Like many modern programming languages, Ada allows subprograms to report error conditions using
exceptions. When an error condition is detected an exception is raised. The exception propagates
to the dynamically nearest handler for that exception. Once the handler has executed the
exception is said to have been handled and execution continues after the handler. Exceptions are
not resumable. This model is largely the same as that used by C++ although the terminology is
different. However, unlike C++, Ada exceptions are not ordinary objects with ordinary types. In
fact, Ada exceptions don't have a type at all and exist outside the language's type system.

Exceptions are typically defined in a package specification. They are then raised in the package
body under appropriate conditions and handled by the package users. The following example shows
part of a specification for a |Big_Number| package that supports operations on arbitrary
precision integers.

\begin{lstlisting}
package Big_Number is
  type Number_Type is private;

  Divide_By_Zero : exception;

  function "+"(Left, Right : Number_Type) return Number_Type;
  function "-"(Left, Right : Number_Type) return Number_Type;
  function "*"(Left, Right : Number_Type) return Number_Type;
  function "/"(Left, Right : Number_Type) return Number_Type;

private
  -- Not shown.
end Big_Number;
\end{lstlisting}

Notice that Ada, like C++, allows operators to be overloaded. In Ada this is done by defining
functions with names given by the operator names in quotation marks. This package also defines
an exception that will be raised when one attempts to use |Big_Number."/"| with a zero divisor.

The implementation of the division operation might look, in part, as follows:

\begin{lstlisting}
function "/"(Left, Right : Number_Type) return Number_Type is
begin
  -- Assume there is a suitable overloaded “=” operator.
  if Right = 0 then
    raise Divide_By_Zero;
  end if;

  -- Proceed knowing that the divisor is not zero.
end "/";
\end{lstlisting}

\noindent As you might guess, the |raise| statement aborts execution of the function
immediately. A procedure that wishes to use |Big_Number| might try to handle this exception in
some useful way. The following example shows the syntax:

\begin{lstlisting}
procedure Example is
  use type Big_Number.Number_Type;
  X, Y, Z : Big_Number.Number_Type;
begin
  -- Fill Y and Z with interesting values.
  X : = Y / Z;

exception
  when Big_Number.Divide_By_Zero =>
    Put_Line("Big Number division by zero!");

  when others =>
    Put_Line("Unexpected exception!");
end Example;
\end{lstlisting}

\noindent The block delimited by |begin| and |end| can contain a header |exception|. After that
header a series of |when| clauses specify which exceptions the block is prepared to handle. The
special clause |when others| is optional and is used for all exceptions that are otherwise not
mentioned.

If the normal statements in the block execute without exception, control continues after the
block (skipping over the exception clauses). In the previous example the procedure returns at
that point. If an exception occurs in the block, that exception is matched against those
mentioned in the when clauses. If a match is found the corresponding statements are executed and
then control continues after the block. If no match is found, the block is aborted and the next
dynamically enclosing block is searched for a handler instead. If you are familiar with C++ or
Java exceptions none of this should be surprising.

The previous example also shows an interesting detail related to operator overloading. The
example assumes that there is a context clause of |with Big_Number| (not shown) but no |use|
statement. Thus the division operator is properly named |Big_Number."/"|. Unfortunately it can't
be called using the infix operator notation with that name. There are several ways to get around
this. One could include a |use Big_Number| in the context clause or in the procedure's
declarative part. However that would also make all the other names in package |Big_Number|
directly visible and that might not be desired. An alternative is to introduce a local function
named |"/"| that is just a renaming (essentially an alias) of the function in the other package.
As we have seen Ada allows such renaming declarations in many situations, but in this case,
every operator function would need a corresponding renaming and that could be tedious.

Instead the example shows a more elegant approach. The |use type| declaration tells the compiler
that the primitive operations of the named type should be made directly visible. I will discuss
primitive operations in more detail in the section on object oriented programming. In this case,
the operator overloads that are declared in package |Big_Number| are primitive operations. This
method allows all the operator overloads to be made directly visible with one easy statement,
and yet does not make every name in the package directly visible.

The Ada language specifies four predefined exceptions. These exceptions are declared in package
|Standard| (recall that package |Standard| is effectively built into the compiler) and thus
directly visible at all times. The four predefined exceptions with their use are described as
follows:

\begin{itemize}

\item |Constraint_Error|. Raised whenever a constraint is violated. This includes going outside
  the bounds of a subtype (or equivalently the allowed range of an array index) as well as
  various other situations.

\item |Program_Error|. Raised when certain ill formed programs that can't be detected by the
  compiler are executed. For example, if a function ends without executing a return statement,
  |Program_Error| is raised.

\item |Storage_Error|. Raised when the program is out of memory. This can occur during dynamic
  memory allocation, or be due to a lack of stack space when invoking a subprogram. This can
  also occur during the elaboration of a declarative part (for example if the dynamic bounds
  on an array are too large).

\item |Tasking_Error|. Raised in connection with certain tasking problems.

\end{itemize}

Most of the time the predefined exceptions are raised automatically by the program under the
appropriate conditions. It is legal for you to explicitly raise them if you choose, but it is
recommended that you define your own exceptions for your code. Note that the Ada standard
library defines some additional exceptions as well. Those exceptions are not really predefined
because, unlike the four above, they are not built into the language itself.

