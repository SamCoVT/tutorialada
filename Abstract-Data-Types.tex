\section{Abstract Data Types}

Often it is desirable to hide the internal structure of a type from its users. This allows the
designer of that type to change its internal structure later without affecting the code that
uses that type. Ada supports this concept by way of private types. A type can be declared as
private in the visible part of a package specification. Users can then create variables of that
type and use the built in assignment and equality test operations on that type. All other
operations, however, must be provided in the package where the type is defined. Subprograms in
the same package as the type have access to the type's internal representation. The following
example shows a hypothetical package that provides a type representing calendar dates.

\begin{lstlisting}
package Example is
   type Date is private;

   function Make(Year, Month, Day : Integer) return Date;
   function Get_Year(Today : Date) return Integer;
   -- etc...
   function Day_Difference(Future, Past : Date) return Integer;
   -- etc...

private
   type Date is
      record
         Y : Integer;
         M : Integer;
         D : Integer;
      end record;
end Example;
\end{lstlisting}

Note that in a more realistic version of this package, one might introduce different types for
year, month, and day values in order to prevent programs from accidentally mixing up those
concepts. However, in the interest of brevity I do not show that in this example. The type
|Date| is declared private in the visible part of the specification. It is then later fully
defined as a record in the private part.

Theoretically the full view of a type should not be in the specification at all; it is part of
the package's implementation and thus should only be in the package body. However, Ada requires
that private types be fully defined in package specifications as a concession to limited
compiler technology. When the compiler translates code that declares variables of the private
type, it needs to know how much memory to set aside for such variables. Thus while the
programmer is forbidden to make use of the private type's internal structure, the compiler needs
to do so. Note that some programming languages go further than Ada in this regard and move all
such private information out of the interface definition. However, there is generally an
execution performance penalty involved in doing this. In Ada it is possible to simulate such a
feature using the so called ``pimpl idiom.'' This method involves access types and controlled
types and so is not described further here.

Note also that the full view of a private type does not need to be a record type. Although
records are common in this situation, private types can be implemented as arrays or even just
simple scalars like |Integer| or |Float|.

Private types do allow assignment and equality tests. However, in some cases it is desirable to
disallow those things. Ada makes this possible using limited private types. A declaration such
as:

\begin{lstlisting}
type Date is limited private;
\end{lstlisting}

\noindent tells the compiler to disallow built in assignment and equality tests for the
specified type. This does not mean assignment is impossible; it only means that the package
author must now provide a procedure to do assignment---if support for assignment is desired.
Presumably that procedure would take whatever steps were necessary to make the assignment work
properly. Calendar dates are not good candidates to make into a limited type. The component-wise
assignment of one date to another is exactly the correct way to assign dates. Thus the built in
record assignment operation is fine. Similar comments apply to the built in test for equality
operation. However, types that represent resources outside the computer system (for example a
network connection) can't reasonably be assigned. They should be declared limited private
instead.

