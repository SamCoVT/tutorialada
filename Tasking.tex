\section{Tasking}

Many programs can be naturally described as multiple, interacting, concurrent threads of
execution. Programming environments that provide direct support for concurrency are thus very
useful. Such support can be offered by the operating system or by the programming language or
some combination of the two.

The classic way in which operating systems support concurrency is by allowing multiple
independent processes to run simultaneously. This method has the advantage of offering good
isolation between the processes so that if one crashes the others are not necessarily affected.
On the other hand, communication between processes tends to have high overhead. Modern operating
systems also usually allow programs to be written that have multiple threads of control
executing in the same process. This \newterm{thread level concurrency} is harder to program
correctly but has reduced overhead as compared to the older style \newterm{process level
  concurrency}.

In some environments offering thread level concurrency the programmer must invoke subprograms in
a special library to create and synchronize threads. Such an approach requires relatively little
support from the programming language but it tends to be error prone. Another approach is to
build support for thread level concurrency into the programming language itself. This allows the
compiler to take care of the low level details of thread management, freeing the programmer to
focus on other aspects of the program's design. Ada uses this second approach and supports
concurrent programming as a language feature.

The unit of execution in Ada is called a \newterm{task}. The main program executes in a task
called the \newterm{environment task}. You can create additional tasks as appropriate to meet
your application's needs. Like a package each task has a specification and a body. In the
simplest case a task specification only needs to declare that the task exists. The following
example illustrates the basic syntax.

\begin{lstlisting}
with Ada.Text_IO;
with Helper;

procedure Main is

  -- Specification of nested task.
  task Nag;

  -- Body of nested task.
  task body Nag is
  begin
    for I in 1 .. 100 loop
      Ada.Text_IO.Put_Line("Hello");
      delay 10.0;
    end loop;
  end Nag;

begin
  Helper.Something_Useful;
end Main;
\end{lstlisting}

In this example a task |Nag| is both specified and defined in the declarative part of the main
program. The task simply prints ``Hello'' one hundred times with a 10 second delay between each
line of output. While the task does this important work, the main program simultaneously
executes the useful function of the program.

It is important to understand that the task starts executing automatically as soon as the
enclosing subprogram begins. It is not necessary to explicitly start the task. Furthermore the
enclosing subprogram will not return until the task has completed. Care must be taken to ensure
that the task eventually ends. If the task never ends the enclosing subprogram will never
return.

Because the task is nested inside another program unit it has access to the local variables and
other entities declared in the enclosing unit above the task's definition. This gives a way for
the task to share information with the enclosing unit but beware that sharing data in this
manner is difficult and error prone. As I will describe shortly Ada provides much more robust
ways for tasks to communicate.

The example above shows a task nested inside a procedure. Tasks can also be nested inside
functions or even inside each other. This allows you to create a task to assist in the execution
of any subprogram without anyone outside your subprogram being aware that tasks are involved.
You can also create tasks inside a package body that support the operation of that package. Be
aware, however, that if you do create a task inside a library package body you need to arrange
for that task to eventually end or else the program will never terminate.

\textit{Finish me\ldots}

