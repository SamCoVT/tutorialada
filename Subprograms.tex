\section{Subprograms}

Unlike C family languages, Ada distinguishes between procedures and functions. Specifically,
functions must return a value and must be called as part of a larger expression. Procedures
never return a value (in the sense that functions do) and must be called in their own
statements. Collectively procedures and functions are called subprograms in situations where
their differences don't matter.

Subprograms can be defined in any declarative part. Thus it is permissible to nest subprogram
definitions inside each other. A nested subprogram has access to the parameters and local
variables in the enclosing subprogram that are defined above the nested subprogram. The scope
rules are largely what you would expect. Listing~\ref{lst:prime-program2} shows a variation of
the prime number checking program first introduced on page~\pageref{lst:prime-program}. This
variation introduces a nested function |Is_Prime| that returns |True| if its argument is a prime
number.

\begin{figure}[tbhp]
\begin{lstlisting}[
  frame=single,
  xleftmargin=0in,
  caption={Prime Checking Program, Version 2},
  label=lst:prime-program2]
with Ada.Text_IO;         use Ada.Text_IO;
with Ada.Integer_Text_IO; use Ada.Integer_Text_IO;

procedure Prime2 is
   Number : Positive;

   -- This function returns True if N is prime; False otherwise.
   function Is_Prime(N : Positive) return Boolean is
   begin
      for I in 2 .. (N - 1) loop
         if N mod I = 0 then
            return False;
         end if;
      end loop;
      return True;
   end Is_Prime;

begin
   Put("Enter an positive integer: ");
   Get(Number);
   if Number < 2 then
      Put("The value "); Put(Number, 0); Put_Line(" is bad.");
   else
      Put("The value "); Put(Number, 0);
      if Is_Prime(Number) then
         Put_Line(" is prime.");
      else
         Put_Line(" is not prime.");
      end if;
   end if;
end Prime2;
\end{lstlisting}
\end{figure}

In this case |Is_Prime| has been declared to take a parameter |N| of type |Positive|. It could
have also accessed the local variable |Number| directly. However, it is usually better style to
pass parameters to subprograms where it makes sense. In this case I also wanted to illustrate
the syntax for parameter declarations. Note that because |Is_Prime| is a function, a return type
must be specified and it must be called in the context of an expression, such as in the
condition of an |if| statement as is done in Listing~\ref{lst:prime-program2}.

Procedure declarations are similar except that no return type is mentioned. Also when a
procedure is called it must not be part of an expression, but rather stand by itself as a single
statement. See the calls to procedures |Put| and |Put_Line| in the examples so far.

The program above also shows the form of a comment. Ada comments start with |--| and run to the
end of the line. As you know, it is good to include comments in your program. However, the
examples in this tutorial do not include many comments in order to save space and because the
programs are explained in the text anyway.

Subprogram parameters can also have one of three possible ``modes.'' The listing below shows the
definition of a procedure that illustrates this. Notice that a semicolon is used to separate
parameter declarations and not a comma as is done in C family languages.

\begin{lstlisting}
procedure Mode_Example
  (X : in Integer; Y : out Integer; Z : in out Integer) is
begin
  X := 1;      -- Error. Can't modify an in parameter.
  Y := X;      -- Okay. Can read X and modify Y.
  Z := Z + 1;  -- Okay. Can read and write an in out parameter.
end Mode_Example;
\end{lstlisting}

The first parameter declared above has mode |in|. Parameters with this mode are initialized with
the argument provided by the caller, but treated as constants inside the procedure. They can be
read, but not modified.

The second parameter has mode |out|. Parameters with this mode are effectively in an
uninitialized state when the procedure begins, but they can be used otherwise as ordinary
variables inside the procedure. In particular |out| parameters can be read provided you first
write a value into them\footnote{Certain composite and aggregate objects, such as arrays, have
  components that are initialized even when the data stored in the object is not. For example,
  the bounds on arrays used as out parameters can always be read.}. Whatever value is assigned
to an |out| parameter by the time the procedure ends is sent back to the calling environment.

The third parameter has mode |in out|. Parameters with this mode are initialized with the
argument provided by the caller, and can also be modified inside the procedure. The changed
value is then returned to the caller to replace the original value. Keep in mind that, unlike in
C, modifications to parameters in Ada (when allowed by the parameter mode) affect the arguments
used when the procedure is called.

The mode |in| is the default. In versions of Ada prior to Ada 2012, functions parameters could
also only have mode |in|, and thus it is common to leave the mode specification off when
defining function parameters. Modern Ada allows function parameters with any of the three modes.
My recommendation, however, is to always specify the mode when declaring procedure parameters
but accept the default of |in|, except under special circumstances, when declaring function
parameters. Functions with parameters of mode |out| or |in out| should be rare. Reasoning about
a program where functions modify their arguments can be difficult.

Like C++, Ada also allows you to define default values for subprogram parameters. This is
accomplished by initializing the parameter (using the |:=| assignment symbol) when the parameter
is declared. If no argument is provided for that parameter when the subprogram is called the
default value is used instead.

Ada also allows you to call subprograms using \newterm{named parameter associations}. This
method allows you to associate arguments with the parameters in any order, and it can also
greatly improve the readability of your code---particularly if you've chosen good names for the
parameters. The listing below shows how the procedure defined above might be called. Assume that
the variables |Accumulator|, |N|, and |Output| have been previously declared as integers.

\begin{lstlisting}
Mode_Example(Z => Accumulator, X => N+15, Y => Output);
\end{lstlisting}

\noindent Notice that the order in which the arguments are provided is not the same as the order
in which the parameters are declared. When named association is used, the order is no longer
significant. Notice also that you can use any expression as the argument associated with an |in|
parameter. However, |out| and |in out| parameters must be associated with variables and not
arbitrary expressions since putting a value into the result of an expression doesn't really make
sense. The result of an expression is an anonymous temporary object, and there is no way for you
to access its value.

\subsection*{Exercises}

\begin{enumerate}
\item Write a procedure |Count_Primes| that accepts a range of positive integers and returns the
  number of primes, the smallest prime, and the largest prime in that range. Your implementation
  should use |out| parameters to return its results. It should also make use of the |Is_Prime|
  function defined earlier. Wrap your procedure in a main program that accepts input values from
  the user and outputs the results. Use named parameter association when calling |Count_Primes|.

\item Implement |Count_Primes| from the previous question as a function (or a collection of
  functions) instead. What are the advantages and disadvantages of each approach? After reading
  the following section on records in Ada, implement |Count_Primes| to return a record
  containing the three result values. What are the advantages and disadvantages of this
  approach?
\end{enumerate}

