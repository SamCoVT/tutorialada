\section{Command Line Arguments}

To access command line arguments, a program needs to use the |Ada.Command_Line| package. The
number of arguments (not including the program name) can be retrieved by calling the function
|Ada.Command_Line.Argument_Count|. If no command line arguments are given, |Argument_Count| will
return 0.

Once the number of arguments is known, a program can access the individual command line
arguments by calling the function |Ada.Command_Line.Argument(n)| where |n| is a Positive (starts
at 1) number indicating which argument is desired. This function returns a |String| that
contains the command line argument. Any conversion (eg. to a numerical or enumerated type) will
have to be done by the programmer.

Unlike C based languages, there is no argument 0 to access the name of the program. Instead, Ada
provides the function |Ada.Command_Line.Command_Name| which returns a string containing the name
of the program.

The program in Listing~\ref{lst:commandline-program}  prints the program name and its arguments.

\begin{figure}[tbhp]
\begin{lstlisting}[
    frame=single,
    xleftmargin=0in,
    caption={Command Line Argument Program},
    label=lst:commandline-program]
with Ada.Text_IO; use Ada.Text_IO;
with Ada.Command_Line;

procedure Commandline_Demo is
   -- Allow use of the shorter ACL. instead of Ada.Command_Line.
   package ACL renames Ada.Command_Line;
begin
   -- Print the program name and number of arguments.
   Put_Line("Program Name : " & ACL.Command_Name);
   Put_Line("Number of arguments : " & Natural'Image(ACL.Argument_Count));

   -- Print all of the arguments in the order they were given.
   Put_Line("Arguments :");
   for I in 1..ACL.Argument_Count loop
      Put_Line(ACL.Argument(I));
   end loop;
end Commandline_Demo;
\end{lstlisting}
\end{figure}

