\section{Control Structures}
\label{sec:control-structures}

Ada contains all the usual control structures you would expect in a modern language. The program
in Listing~\ref{lst:prime-program} illustrates a few of them, along with several other features.
This program accepts an integer from the user and checks to see if it is a prime number.

\begin{figure}[tbhp]
\begin{lstlisting}[
  frame=single,
  xleftmargin=0in,
  caption={Prime Checking Program},
  label=lst:prime-program]
with Ada.Text_IO;         use Ada.Text_IO;
with Ada.Integer_Text_IO; use Ada.Integer_Text_IO;

procedure Prime is
   Number : Integer;
begin
   Put("Enter an integer: ");
   Get(Number);
   if Number < 2 then
      Put("The value "); Put(Number, 0); Put_Line(" is bad.");
   else
      Put("The value "); Put(Number, 0);
      for I in 2 .. (Number - 1) loop
         if Number rem I = 0 then
            Put_Line(" is not prime.");
            return;
         end if;
      end loop;
      Put_Line(" is prime.");
   end if;
end Prime;
\end{lstlisting}    
\end{figure}

The program declares a local variable named |Number| of type |Integer| in its declarative part.
Notice that the type appears after the name being declared (the opposite of C family languages),
separated from that name with a colon.

Procedure |Get| from package |Ada.Integer_Text_IO| is used to read an integer from the console.
If the value entered is less than two an error message is displayed. Notice that there are two
different |Put| procedures being used: one that outputs a string and another that outputs an
integer. Like C++, Java, and some other modern languages, Ada allows procedure names to be
overloaded distinguishing one procedure from another based on the parameters. In this case the
two different |Put| procedures are also in different packages but that fact is not immediately
evident because of the |use| statements.

If the value entered is two or more, the program uses a |for| loop to see if any value less than
the given number divides into it evenly (that is, produces a zero remainder after division, as
calculated with the |rem| operator). Ada |for| loops scan over the given range assigning each
value in that range to the \newterm{loop parameter variable} (named |I| in this case) one at a
time. Notice that it is not necessary to explicitly declare the loop parameter variable. The
compiler deduces its type based on the type used to define the loop's range. It is also
important to understand that if the start of the range is greater than the end of the range, the
result is an empty range. For example, the range |2 .. 1| contains no members and a loop using
that range won't execute at all. It does not execute starting at two and counting down to one.
You need to use the word |reverse| to get that effect:

\begin{lstlisting}
for I in reverse 1 .. 2 loop
\end{lstlisting}

Notice also that |if| statements require the word |then| and that each |end| is decorated by the
name of the control structure that is ending. Finally notice that a single equal sign is used to
test for equality. There is no ``=='' operator in Ada.

The program in Listing~\ref{lst:vowel-program} counts the vowels in the text at its standard
input. You can provide data to this program either by typing text at the console where you run
it or by using your operating system's I/O redirection operators to connect the program's input
to a text file.

\begin{figure}[tbhp]
\begin{lstlisting}[
  frame=single,
  xleftmargin=0in,
  caption={Vowel Counting Program},
  label=lst:vowel-program]
with Ada.Text_IO;         use Ada.Text_IO;
with Ada.Integer_Text_IO; use Ada.Integer_Text_IO;

procedure Vowels is
   Letter      : Character;
   Vowel_Count : Integer := 0;
   Y_Count     : Integer := 0;
begin
   while not End_Of_File loop
      Get(Letter);
      case Letter is
         when 'A'|'E'|'I'|'O'|'U' |
              'a'|'e'|'i'|'o'|'u' =>
            Vowel_Count := Vowel_Count + 1;
            
         when 'Y'|'y' =>
            Y_Count := Y_Count + 1;
            
         when others =>
            null;
      end case;
   end loop;
   Put("Total number of vowels = "); Put(Vowel_Count); New_Line;
   Put("Total number of Ys = "); Put(Y_Count); New_Line;
end Vowels;
\end{lstlisting}
\end{figure}

This program illustrates |while| loops and |case| structures (similar to C's switch statement).
There are quite a few things to point out about this program. Let's look at them in turn.

\begin{itemize}
\item Variables can be initialized when they are declared. The |:=| symbol is used to give a
  variable its initial value (and also to assign a new value to a variable). Thus like C family
  languages, Ada distinguishes between test for equality (using |=|) and assignment (using
  |:=|). Unlike C family languages you'll never get them mixed up because the compiler will
  catch all incorrect usage as an error.

\item In this program the condition in the |while| loop involves calling the function
  |End_Of_File| in package |Ada.Text_IO|. This function returns |True| if the standard input
  device is in an end-of-file state. Notice that Ada does not require (or even allow) you to use
  an empty parameter list on functions that take no parameters. This means that function
  |End_Of_File| looks like a variable when it is used. This is considered a feature; it means
  that read-only variables can be replaced by parameterless functions without requiring any
  modification of the source code that uses that variable.

\item The program uses the logical operator |not|. There are also operators |and| and |or| that
  can be used in the expected way.

\item The |case| statement branches to the appropriate |when| clause depending on the value in
  the variable |Letter|. The two |when| clauses in this program show multiple alternatives
  separated by vertical bars. If any of the alternatives match, that clause is executed. You can
  also specify ranges in a |when| clause such as, for example, |when 1 .. 10 => etc|.

\item Unlike C family languages there is no ``break'' at the end of each |when| clause. The flow
  of control does \emph{not} fall through to the next |when| clause.

\item Ada has what are called \newterm{full coverage rules}. In a |case| statement you must
  account for every possible value that might occur. Providing |when| clauses for just the
  vowels would be an error since you would not have fully covered all possible characters (the
  type of |Letter| is |Character|). In this program I want to ignore the other characters. To do
  this, I provide a |when others| clause that executes the special |null| statement. This lets
  future readers of my program know that I am ignoring the other characters intentionally and
  it's not just an oversight.

\item |New_Line| is a parameterless procedure in package |Ada.Text_IO| that advances the output
  to the next line. More accurately, |New_Line| is a procedure with a default parameter that can
  be used to specify the number of lines to advance, for example: |New_Line(2)|.
\end{itemize}

\subsection*{Exercises}

\begin{enumerate}
\item The prime number program in Listing~\ref{lst:prime-program} is not very efficient. It can
  be improved by taking advantage of the following observation: If |Number| is not divisible by
  any value less than |I|, then it can't be divisible by any value greater than |Number/I|. Thus
  the upper bound of the loop can be reduced as |I| increases. Modify the program to use this
  observation to improve its performance. Note: You will have to change the |for| loop to a
  |while| loop (try using a |for| loop first and see what happens).

\item Modify the prime number program again so that it loops repeatedly accepting numbers from
  the user until the user types -1. You can program an infinite loop in Ada as shown below. Have
  your program print different error messages for negative numbers (other than -1) than for the
  values zero and one. Use an |if| \ldots\ |elsif| chain (note the spelling of |elsif|). Write a
  version that uses a |case| statement instead of an |if| \ldots\ |elsif| chain. You can use
  |Integer'First| in a range definition to represent the first (smallest) |Integer| value the
  compiler can represent.

  \begin{lstlisting}[escapechar=\@]
  loop
     ...
     exit when @\textit{condition}@
     ...
  end loop;
  \end{lstlisting}

\item \textit{Challenging}. The vowel counting program in Listing~\ref{lst:vowel-program} counts
  `Y' separately because in English `Y' is sometimes a vowel and sometimes not. Modify the
  program so that it adds occurrences of `Y' to the vowel counter only when appropriate.
\end{enumerate}

